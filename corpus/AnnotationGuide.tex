% \begin{preamble}
\documentclass[11pt]{article}
\usepackage[margin=1in]{geometry}
\usepackage[utf8]{inputenc}
\usepackage{palatino}
\usepackage{color}
% \end{preamble}


\definecolor{Purple}{RGB}{255,10,140}
\newcommand{\jd}[1]{\textcolor{Purple}{[jd: #1]}}


\begin{document}

% Title
\noindent \huge \textbf{Wh-Questions Tgrep2 Corpus Annotation Guide}\\

\noindent \Large Morgan Moyer \hfill Version: October 22, 2020
% Endtitle 
\normalsize

\bigskip
\hrule 
\bigskip

\section{Macros}

\begin{enumerate}
    \item \textbf{Disfluencies @ DISFL:}\\
    /EDITED$|$UH$|$PRN$|$-UNF/
    
    \item \textbf{Wh-Node @ WH:}\\
    /WP$|$WRB$|$WDT/ (The word-level wh-node)\\
    !$<$ /$\string^$t$|\string^$T/ (that doesn't termininate in a \textsf{that}/\textsf{That})\\
    !$>\!\!>$ @DISFL (isn't dominated by a disfluency)\\
    $>\!\!>$ /$\string^$S/ (but is dominated by an S node)\\
    !$>$ /$\string^$N$|\string^$A/ (isn't parented by any NPs or ADV/ADJPs)\\
    $>\!\!>$ /\string^WH/
\end{enumerate}


\section{Categorical Variables}

\begin{enumerate}
    \item \textbf{WhPhraseType}: \\
    Returns ``complex'' if @WH has a sister (e.g., \textsf{which children}); returns ``monomorphemic'' otherwise (e.g., \textsf{who}).
        \begin{enumerate}  
            \item Complex: @WH [\$ /$\string^$N$|\string^$D$|\string^$J$|\string^$RB$|\string^$RP$|\string^$PP/
            $|$ $>\!\!>$ (/$\string^$WH/ \$ /$\string^$AD$|\string^$JJ$|\string^$NP$|\string^$PP/)]

            \item Monomorphemic: @WH [!\$ /$\string^$N$|\string^$D$|\string^$J$|\string^$RB$|\string^$RP$|\string^$PP/
            $|$ !$>\!\!>$ (/$\string^$WH/ \$ /$\string^$AD$|\string^$JJ$|\string^$NP$|\string^$PP/)]
        \end{enumerate}

    \item \textbf{ModalPresent}: \\
    Returns ``yes'' if there is a modal auxiliary (\textsf{can}, \textsf{could}, \textsf{shall}, \textsf{should}, \textsf{may}, \textsf{might}, \textsf{must}) in the clause that is sister to the WH-phrase node that dominates the @WH (\textsf{where we can find coffee}); Returns ``no'' if there isn't any. 
        \begin{enumerate}
            \item Yes: @WH $>\!\!>$ (/$\string^$WH/ $>\!\!>$ /$\string^$S/ \$ (/S$|$SQ/ $<\!\!<$ /MD/))
            \item No: @WH !$>\!\!>$ (/$\string^$WH/  $>\!\!>$ /$\string^$S/ \$ (/S$|$SQ/ $<\!\!<$ /MD/))
        \end{enumerate}
        
    \item \textbf{HaveNeed}:\\
    Returns 'yes' if there's a \textsf{have to} or \textsf{need to} construction. Returns 'no' otherwise.
        \begin{enumerate}
            \item Yes: @WH $>\!\!>$ (/$\string^$WH/ \$ (/$\string^$S/ $<\!\!<$ (/$\string^$VP/ $<\!\!<$ /need$|$needs$|$needed$|$have$|$has$|$had$|$'ve$|$'d/) $<\!\!<$ /TO/ ))

            \item No: @WH !$>\!\!>$ (/$\string^$WH/ \$ (/$\string^$S/ $<\!\!<$ (/$\string^$VP/ $<\!\!<$ /need$|$needs$|$needed$|$have$|$has$|$had$|$'ve$|$'d/) $<\!\!<$ /TO/) )

        \end{enumerate}
    %     \jd{
    %     \begin{itemize}
    %         \item the problem with these patters, ie why it's consistently claiming there are no cases of modals, is that you're asking the modal to directly follow the wh-word, but most cases are such that there is other material intervening, including in the example that you list. once i realized this, i stopped annotating the cases for missing modal, so only some of them are marked.
    %         \item i did also see a case -- 122760:31 ``what would we do without it" -- that, even though it does have the modal directly following the wh-word, still returned "no". mysterious; figure out what's happening.
    %         \item there was a case -- 165374:11 ``i know what needs to be done" -- that seems like it should return "yes" because of the "needs to" -- can you include "need/needs/needed to" and "have/has/had to"? others like this?
    %     \end{itemize}
    % }
    \item \textbf{Finite}\\
    Returns ``finite'' if the question contains a tensed (=non-modal) verb (\textsf{where they found coffee}); returns ``infinite'' if the question contains an infinitival clause (\textsf{where to find coffee})
        \begin{enumerate}
            \item \textbf{Finite}: \\
            @WH $>\!\!>$ (/$\string^$WH/ !\$ (/$\string^$S/ !\$ /$\string^$VB/ $<$ (/$\string^$NP/ $<$ /-NONE-/ \$. (/$\string^$VP/ $<$ /TO/))))
            \item \textbf{Infinitival}: \\
            @WH $>\!\!>$ (/$\string^$WH/ $ (/$\string^$S/ !$ /$\string^$VB/ $<$ (/$\string^$NP/ $<$ /-NONE-/ \$. (/$\string^$VP/ $<$ /TO/))))
        \end{enumerate}
    
    \item \textbf{QuestionType}\\
    Categorizes the kind of wh-question. We are mostly interested in Root (\textsf{Where can we find coffee?}) and Embedded (Dana knows where we can find coffee), and possibly Embedded Adjuncts. Embedded Adjuncts may look on the surface like Embedded Clauses, the difference only being whether the wh-clause is a complement to a verb or an adjunct. 

    \item[] \textbf{\textsc{Critical QuestionTypes}}
        \begin{enumerate}
            \item \textbf{Root}\\
            Wh-clauses that have interrogative illocutionary force (e.g., \textsf{Where can I find coffee?}). Are not complements to adjectives/verbs or children of Ns\\
            @WH \\
            $>\!\!>$ (/$\string^$SQ$|\string^$SBARQ/ [!\$ /$\string^$VB$|\string^$JJ$|\string^$BE/ $|$ !$>$ /$\string^$N/]) \\
            !$>\!\!>$ /SBAR-NOM/

            \item \textbf{Embedded}\\ 
            Wh-clauses that are complements (sisters) to a verb (VB) or an adjective (JJ), OR that are embedded in PP which are complements of verbs or adjectives.

            @WH \\
            $>\!\!>$ ($/\string^$SBAR/ [\$ /$\string^$VB$|\string^$JJ$|\string^$BE/] $|$ [$>$ /$\string^$PP/ \$ /$\string^$JJ$|\string^$VB$|\string^$BE/]) \\
            !$>\!\!>$ (/SBAR-NOM/ [!\$ /$\string^$V $|$\string^JJ$|\string^$BE/] $|$ [!$>$ (/$\string^$PP/ \$ /$\string^$VB$|\string^$JJ$|\string^$BE/)])

            \item \textbf{EmbAdjunct}\\
            Wh-clauses that are not complements to a verb/adjective, but linearly follow the verb/adjective (i.e., that look like embedded clauses from a linear perspective).\\
            @WH\\ 
            $>\!\!>$ (/$\string^$SBAR/ \$ /$\string^$VP$|\string^$AD/ !\$ /$\string^$VB$|\string^$JJ$|\string^$BE/)
            
            \end{enumerate}
        \item[] \textbf{\textsc{Non-Critical Question-Types}}\\
        These are tags meant to weed out spurious root/embedded questions.

        \begin{enumerate}
            \item \textbf{Relative}\\
            The Wh-clause modifies a noun. (\textsf{Dana saw the boy who ate Captain Crunsh}).
            @WH \\
            $>\!\!>$ (/$\string^$S/ [$>$ /$\string^$N/ $|$ $>$ (/$\string^$P/ $>$ /$\string^$N/)]) \\
            !$>\!\!>$ (/SBAR-NOM/ [\$ /$\string^$VB$|\string^$JJ$|\string^$BE/] $|$ [$>$ (/$\string^$PP/ \$ /$\string^$VB$|\string^$JJ$|\string^$BE/)])

            \item \textbf{Adjunct}\\ 
            Wh-clauses that are neither complements of verbs/adjectives, nor the children of PPs that are complements of verbs/adjectives
            @WH\\
            % i don't know why but i had to put a phantom space before the bracket
            % to allow the bracket to start the new line
            {}[$>\!\!>$ (/$\string^$SBAR/ [$>$ (/$\string^$PP/ !\$ /$\string^$VB$|\string^$JJ$|\string^$BE/)] $|$ [!\$ /$\string^$VB$|\string^$JJ$|\string^$BE/])]

            \item \textbf{Nominalized}\\
            (this is now redundant with the above)
            Nominalized wh-clauses that are not complements.\\
            @WH \\
            $>\!\!>$ (/SBAR-NOM/ [!\$ /$\string^$VB$|\string^$JJ$|\string^$BE/] $|$ [!$>$ (/$\string^$PP/ \$ /$\string^$VB$|\string^$JJ$|\string^$BE/)])

            \item \textbf{Subject}\\
            Wh-clauses that are subjects.\\
             @WH \\
            {}[ $>\!\!>$ /SBAR-SBJ/ $|$ $>\!\!>$ (/$\string^$S/ $>\!\!>$ /TOP/)]

            \item \textbf{Fragment}\\
            Wh-clauses that are fragments.
             @WH \\
            $>\!\!>$ (/$\string^$SBAR/ $>$ /FRAG/)
            
            \item \textbf{Exclam} \\
            Wh-clauses that are exclamatives (don't have a verb), e.g., \textsf{what a meal!}
            @WH\\ 
            $>\!\!>$ (/$\string^$S/ !$<\!\!<$ /$\string^$V/) \\
            !$>\!\!>$ /$\string^$SBAR-NOM/
        \end{enumerate}
    \item \textbf{DegreeQ}\\
    Returns `yes' if the wh-clause is a degree-question (\textsf{how many cups of salt do we need?}); returns `no' otherwise.
        \begin{enumerate}
            \item Yes: @WH \$ /$\string^$JJ$|\string^$RB/
            \item No: @WH !\$ /$\string^$JJ$|\string^$RB/
        \end{enumerate}
    \item \textbf{WhAll}\\
    Returns `yes' if the wh-phrase contains an \textsf{all}; returns `no' otherwise.
        \begin{enumerate}
            \item Yes: @WH $>\!\!>$ (/$\string^$WH/ $<\!\!<$ (/$\string^$NP/ $<$ (/$\string^$DT/ $<$ /all/) !$<$ /$\string^$N/))
        \end{enumerate}
    \item \textbf{SubjectAux}\\
    Returns `yes' if the wh-clause contains subject-aux-inversion.
        \begin{enumerate}
            \item Yes: @WH $>\!\!>$ (/$\string^$SBAR/ $<\!\!<$ /$\string^$SQ/)
        \end{enumerate}
    
\end{enumerate}

\section{String Variables}
\begin{enumerate}
    \item \textbf{MatVerb}: Print all the matrix predicates who have the wh-clause as complement.\\
    @WH $>\!\!>$ (/$\string^$S/ \$ /$\string^$VB$|\string^$JJ$|\string^$BE/=print)

    \item \textbf{MatVerbPart}: Print the matrix predicates who have wh-clause as complement to PP (e.g., \textsf{surprised by}, \textsf{agree on}, \textsf{depend on}).\\
    @WH $>\!\!>$ (/$\string^$S/ (/$\string^$PP/ \$ /$\string^$VB$|\string^$JJ$|\string^$BE/=print))

    \item \textbf{Wh}: Print the wh-word terminal node.\\
    @WH $<$ /$\string^$w$|\string^$h/=print

    \item \textbf{Modal}: Print any modals following the @WH node.\\
    @WH $>\!\!>$ (/$\string^$WH/ $>\!\!>$ /$\string^$S/ \$ (/S$|$SQ/ $<\!\!<$ /MD/=print))
    
    \item \textsf{HaveNeed}: Print any \textsf{have to} / \textsf{need to} modal constructions
    @WH $>\!\!>$ (/$\string^$WH/ \$ (/$\string^$S/ $<\!\!<$ (/$\string^$VP/ $<\!\!<$ /need$|$needs$|$needed$|$have$|$has$|$had$|$'ve$|$'d/=print) $<\!\!<$ /TO/ ))

    \item \textbf{Verb1}: Print the first verb in the wh-clause. \\
    @WH $>\!\!>$ (/$\string^$S/ [$<\!\!<$ /$\string^$VP/ $<\!\!<$ /$\string^$VB$|\string^$JJ$|\string^$BE/=print])

    \item \textbf{Verb2}: Print the second verb/predicate in the wh-clause.\\
    @WH $>\!\!>$ (/$\string^$S/ [$<\!\!<$ /$\string^$VP/ [$<$ /$\string^$VP/ $<$ /$\string^$VB$|\string^$JJ$|\string^$BE/=print]])
    
    \item \textbf{Verb3}: Print the third verb/predicate in the wh-clause.\\
    @WH $>\!\!>$ (/$\string^$S/ [$<\!\!<$ /$\string^$VP/ [$<\!\!<$ /$\string^$VP/ [$<$ /$\string^$VP/ $<$ /$\string^$VB$|\string^$JJ$|\string^$BE/=print]]])
    
    \item \textbf{DeterminerSubject}: Print any determiners in the wh-clause NP-SBJ.\\
    @WH $>\!\!>$ (/$\string^$S/$<\!\!<$ (/DT/=print $>\!\!>$ /$\string^$NP-SBJ/))
    
    \item \textbf{DeterminerNonSubject}: Print any determiners in the wh-clause VP.\\
    @WH $>\!\!>$ (/$\string^$S/ $<\!\!<$ (/DT/=print $>\!\!>$ /$\string^$VP/))

    \item \textbf{FullWhPhrase}: Print the full WH-Phrase.\\
    @WH $>\!\!>$ (/$\string^$WH/=print \$ /S$|$SQ/ $>\!\!>$ /$\string^$S/ $>\!\!>$ /$\string^$N$|\string^$J$|\string^$RB$|\string^$PP$|\string^$A/)

    \item \textbf{WhParse}: Print the Wh-phrase with POS.

    \item \textbf{Sentence}: Print the full sentence in which the @WH node appears.\\
    @WH $>\!\!>$ ($\ast$=print !$>$ $\ast$)

    \item \textbf{Question}: Print the full wh-clause.\\
    @WH $>\!\!>$ /$\string^$S/=print

    \item \textbf{QuantifiedPredicate}: Print any quantifiers in the wh-clause predicate.\\
    @WH \\
    $>\!\!>$ (/$\string^$WH/ $>$ (/$\string^$S/ $<\!\!<$ (/$\string^$VP/\\
    $<\!\!<$ /all$|$every$|$some$|$one$|$any$|$most$|$least$|$more$|$most$|$much/=print)))

    \item \textbf{QuantifiedSubject}: Print any quantifiers in the wh-clause subject.\\
    @WH \\
    $>\!\!>$ (/$\string^$WH/ $>$ (/$\string^$S/ $<\!\!<$ (/$\string^$NP-SBJ/\\
    $<\!\!<$ /all$|$every$|$some$|$one$|$any$|$most$|$least$|$more$|$most$|$much/=print)))

\end{enumerate}



\end{document}
